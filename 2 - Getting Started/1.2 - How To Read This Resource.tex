This resource is going to be laid out a little weird, more so for those who have already have some programming background.
\par

For those who are new to programming, I strongly suggest reading through Chapters 3 and 4 thoroughly before continuing with the
rest of this resource as they contain all of the core concepts needed in order to understand some of the higher level concepts
presented in Data Structures and Algorithms.
Once you have completed chapters 3 and 4 please feel free to jump around between sections presented in chapters 5 and 6 as
certain data structures or algorithms might interest you more than others.

\subsection{Keywords}

Throughout this resource some words will be highlighted, colored differently or emphasized in order to stand out.
These words will generally be referring to code examples presented in the chapters:
\\

\begin{tabular}{|l | c | r|}
\hline
Type & Example \\ \hline
Variable & \pigVar{variableName} \\
Functions & \pigVar{functionName} \\
Class Properties & \pigVar{propertyName} \\
Values & \pigVal{"Sample String Value"} \\
Program Output & \pigOut{Console String Output} \\ \hline
\end{tabular}
\\

\par

\emph{Example:}
\\
We assign the value of \pigVal{"Sample String"} to the variable \pigVar{sample} then pass in \pigVar{sample} as a parameter to
the function \pigVar{printValue} which will print: \pigOut{The String Is: Sample String}.
