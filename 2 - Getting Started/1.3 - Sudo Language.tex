For the code examples presented in this resource I am going to be using a sudo language.
The concept behind a sudo language is to be able to present programming concepts in a
language agnostic form so that the concepts can be translated to your language of choice.
\par

So it is great that you have chosen lanauge X to use throughout this resource, but how is the sudo lanauge going to help you out?
Well, lets look at two examples and I will show their implementation in a few different languages.
Hopefully this will help you be able to understand how the language should be tanslates (especially if your language of choice
is one that I use).

\subsection{Example 1}

\begin{lstlisting}[caption={Example 1 - Sudo Code}]
name = "Brett"
if name == "Brett"
    print "Name Is Brett"
else
    print "Name Is Not Brett"
\end{lstlisting}

For this example lets break it down line by line to make sure we know exactly what is going on.

\begin{enumerate}
    \item {Store the value \pigVal{"Brett"} into the variable \pigVar{name}}
    \item {Check if the variable \pigVar{name} is equal to the value \pigVal{"Brett"}}
    \begin{enumerate}
        \setcounter{enumii}{2}
        \item {Print \pigOut{"Name Is Brett"} to the console}
    \end{enumerate}
    \setcounter{enumi}{3}
    \item {Otherwise}
    \begin{enumerate}
        \setcounter{enumii}{4}
        \item {Print \pigOut{"Name Is Not Brett"} to the console}
    \end{enumerate}
\end{enumerate}

As far as programming goes this is a fairly simple process but lets try and translate this example to a
few different languages to see how it is done.

\begin{lstlisting}[language=php,caption={Example 1 - PHP}]
<?php
$name = 'Brett';
if( $name === 'Brett' ){
    echo 'Name Is Brett';
} else{
    echo 'Name Is Not Brett';
}
\end{lstlisting}

\begin{lstlisting}[language=c,caption={Example 1 - C}]
int main{
    char* name = "Brett";
    if( name == "Brett" ){
        printf("Name Is Brett");
    } else{
        printf("Name Is Not Brett");
    }
    return 0;
}
\end{lstlisting}

\begin{lstlisting}[language=python,caption={Example 1 - Python}]
name = "Brett"
if name is "Brett":
    print "Name Is Brett"
else:
    print "Name Is Not Brett"
\end{lstlisting}

\begin{lstlisting}[language=javascript,caption={Example 1 - Node.JS}]
var name = "Brett";
if( name == "Brett" ){
    console.log("Name Is Brett");
} else{
    console.log("Name Is Not Brett");
}
\end{lstlisting}

\begin{lstlisting}[language=java,caption={Example 1 - Java}]
class Example1{
    public static void main( String[] args ){
        String name = "Brett";
        if( name.equals("Brett") ){
            System.out.println("Name Is Brett");
        } else{
            System.out.println("Name Is Not Brett");
        }
    }
}
\end{lstlisting}

Notice that all of the actual examples end up looking the same?
That is the point of using the sudo language, so that we can discuss the core concepts of the lesson
at hand and then those concepts can be directly applied to any language of choice.
\par

Also, notice the Python implementation, it is almost line for line, word for word identical to the sudo language example.

\subsection{Example 2}
Since we have seen a fairly simple example above, lets take a look at a more complicated example.
Do not be afraid if it does not make too much sense right now, but try and notice the similarities
between the sudo language and the actual code examples.



\begin{lstlisting}[caption={Example 2 - Sudo Code}]
class Person
    private name
    
    function getName()
        return this.name
    
    function setName( newName )
        this.name = newName


p = new Person()
p.setName("Brett")


if p.getName() == "Brett"
    print "Name Is Brett"
else
    print "Name Is Not Brett"
\end{lstlisting}

Just like the last one, lets break down this example line by line to fine out whats going on.

\begin{enumerate}
    \item {Create a new class called \pigVar{Person}}
    \begin{enumerate}
        \setcounter{enumii}{1}
        \item {Create a private property \pigVar{name}}
        \setcounter{enumii}{3}
    
        \item {Create a method called \pigVar{getName} that requires no parameters}
        \begin{enumerate}
            \setcounter{enumiii}{4}
            \item {When the function is called return the class property \pigVar{name}}
        \end{enumerate}
        \setcounter{enumii}{6}
    
        \item {Create a method called \pigVar{setName} that takes a single parameter \pigVar{newName}}
        \begin{enumerate}
            \setcounter{enumiii}{7}
            \item {When called set the class property \pigVar{name} equal to the parameter \pigVar{newName}}
        \end{enumerate}
    \end{enumerate}
    
    \setcounter{enumi}{10}
    \item {Create a new \pigVar{Person} object and store it in the variable \pigVar{p}}
    \item {Call \pigVar{p}'s \pigVar{setName} method passing in the value \pigVal{"Brett"}}
    
    \setcounter{enumi}{13}
    \item {Call \pigVar{p}'s \pigVar{getName} method and check if the returned value is equal to \pigVal{"Brett"}}
    \begin{enumerate}
        \setcounter{enumii}{14}
        \item {Print \pigOut{"Name Is Brett"} to the console}
    \end{enumerate}
    
    \setcounter{enumi}{15}
    \item {Otherwsie}
    \begin{enumerate}
        \setcounter{enumii}{16}
        \item {Print \pigOut{"Name Is Not Brett"} to the console}
    \end{enumerate}

\end{enumerate}

Do not worry if this example does not make sense to you, you will be able to understand it well before the end of this resource.
\par

Just like with Example 1, here are some translations of the example.

\begin{lstlisting}[language=php,caption={Example 2 - PHP}]
<?php
class Person{
    private $name;
    
    public function getName(){
        return $this->name;
    }
    
    public function setname( $newName ){
        $this->name = $newName;
    }
}

$p = new Person();
$p->setName('Brett');

if( $p->getName() === 'Brett' ){
    echo 'Name Is Brett';
} else{
    echo 'Name Is Not Brett';
}
\end{lstlisting}

\begin{lstlisting}[language=java,caption={Eample 2 - Java}]
class Person{
    private String name;
    
    public String getName(){
        return this.name;
    }
    
    public void setName( String newName ){
        this.name = newName;
    }
    
    public static void main(String[] args){
        Person p = new Person();
        p.setName("Brett");
        
        if( p.getName() == "Brett" ){
            System.out.println("Name Is Brett");
        } else{
            System.out.println("Name Is Not Brett");
        }
    }

}
\end{lstlisting}

\begin{lstlisting}[language=javascript,caption={Example 2 - Node.JS}]
var Person = function(){}
Person.prototype.getName = function(){
    return this.name;
}
Person.prototype.setName = function( newName ){
    this.name = newName;
}

var p = new Person();
p.setName("Brett");

if( p.getName() == "Brett" ){
    console.log("Name Is Brett");
} else{
    console.log("Name Is Not Brett");
}
\end{lstlisting}

\begin{lstlisting}[language=python,caption={Example 2 - Python}]
class Person:
    def getname( self ):
        return self.name
    def setName( self, newName ):
        self.name = newName

p = Person()
p.setName("Brett");

if p.getName() is "Brett":
    print "Name Is Brett"
else:
    print "Name Is Not Brett"
\end{lstlisting}

This example does a better job of showing how each language can tackle the concepts in a different manner but the core concepts
laid out by the sudo language can still be extrapolated and translated to each individual programming language.
As long as the language supports the concepts.
As you may notice that I left out the implementation of C in this example.
It is because C does not support the use of classes and objects, yes there are ways of completing this example in C using structs but that
is something that you should learn on your own.
\par

So now you have seen a few examples, hopefully enough to give you an idea of how the examples in this resource will be presented.
