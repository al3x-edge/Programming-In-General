Variables are used to represent values by name and allow you to access the original value or manipulate the original value.
For example we would store the integer value 10 into a variable named "a" which would then allow us to refer to "10" by
using the name "a". 

\begin{lstlisting}
a = 10
print a
\end{lstlisting}

In the above example the output would be "10" because the value 10 is stored in the variable "a" and then we access the original value
10 when we print a.

\subsection{Data Types}

Programming languages support different types of data types or different types of values that they can represent in variables.
Some programming languages use multiple different types of values but most of them support the basic types: string, integer
and boolean (true or false).

\begin{lstlisting}
string = "Brett"
integer = 10
boolean = false
\end{lstlisting}

Please keep in mind that each programming language supports different data types and you should research those types to better
understand variables in that language.

\subsection{Operations}

\subsection{Conclusion}
