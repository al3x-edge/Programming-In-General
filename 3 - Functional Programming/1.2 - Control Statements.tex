Control statements are almost exactly as they sound, statements that control our programs.
Well, they control the flow of our code.
With control statements we can change the course of our programs based on various conditions.

\subsection{If Statements}
If statement allow us to execute a given block of code based on a given condition.
There are three main parts to an if statement \pigVal{if}, the \pigVal{conditional} and a \pigVal{code block}.

\begin{lstlisting}[caption={If Statement}]
name = ``brett''

if( name == ``brett'' )
    print ``Name is brett''
\end{lstlisting}

In this simple example the code block \pigOut{print ``Name is brett''} will only execute if the conditional \pigOut{name == ``brett''} is true.
So the output of this code will be \pigOut{Name is brett}.

\begin{lstlisting}[caption={False If Statement}]
name = ``brett''

if( name == ``john'' )
    print ``Name is john''
\end{lstlisting}

In this example there will be no output, this is because the conditional \pigOut{name == ``john''} equates to false so the code block \pigOut{print ``Name is john''} will never get executed.

\subsection{If-Else Statements}
If statements are great and help us execute portions of our code based on the values of other variables, including based on input from users.
But what if the condition of the if statement equates to false?
With if statements we can append an else statement and a block of code to the end of an if statement that will get called if the if statement is false. 

\begin{lstlisting}[caption={If-Else Statement}]
name = ``brett''

if name == ``john'':
    print ``Name is john''
else:
    print ``Name is not john''
\end{lstlisting}

The output of this program will be \pigOut{Name is not john}.
When the program hits the if statement it evaluates the conditional \pigVar{name == ``john''} which evaluates to \pigVal{false}.
Normally the program will continue on its way but since we provided an else statement that gets executed instead.
An If-Else statement allows us to program ``if this then do this, otherwise do this.''

\subsection{Else If Statements}
Ok... wait, we just did If-Else statements not we are doing Else if statements?
Yes, but they are different I swear!
An If-Else statement allows us to execute code regardless of whether a conditional is true or false but with an else if statement we can provide multiple conditionals to an if statements.

\begin{lstlisting}[caption={Else If Statement}]
name = ``brett''

if name == ``john'':
    print ``Name is john''
else if name == ``brett''
    print ``Name is brett''
\end{lstlisting}

See? I told you it was different.
So the output of this program is \pigOut{Name is brett} and this is because when the program gets to the if statement and evaluates it as false, it then continues down the list of conditionals.
This works similar to how the else statement before worked, but this time we are giving the if statement multiple conditionals to check.
We can expand this example by adding more else if statements.

\begin{lstlisting}[caption={Else If Statement 2}]
name = ``brett''

if name == ``john'':
    print ``Name is john''
else if name == ``brett''
    print ``Name is brett''
else if name == ``barbara'':
    print ``Name is barbara''
\end{lstlisting}

Just like the first example this program will output \pigOut{Name is brett}.
This is because when the program gets to \pigVar{name == ``john''} it evaluates to false causing the program to skip to the next conditional \pigVar{name ==''brett''}, which then evaluates to true causing the code block given to execute.
The last conditional \pigVar{name == ``barbara''} will then be skipped and the program will continue past the if statement.
\par

Now... what is we add an else statement to the end of this?
With an if statement we could append an else statement to the end telling it what to do if the conditional failed.
With an else if statement we can also append an else statement to the end telling it what to do if all of the conditionals fail.

\begin{lstlisting}[caption={Else If Else Statement}]
name = ``brett''

if name == ``john'':
    print ``Name is john''
else if name == ``barbara'':
    print ``Name is barbara''
else:
   print ``Well, I`m not sure what your name is''
\end{lstlisting}

This program will output \pigOut{Well, I'm not sure what your name is} because both conditionals, \pigVar{name == ``john''} and \pigVar{name == ``barbara''}, evaluate to false causing the if statement to continue on its merry way.

\subsection{Switch Statements}
A Switch statement is similar to a grouping of If, Else If and Else statements but where the conditional is always a direct comparison to a value.
Switch statements are useful when you have a set number of values to compare a variable against.
For example, the following If statements are a perfect candidate for a switch statement.

\begin{lstlisting}[caption={Switch Statement Candidate}]
name = ``brett''

if name == ``john'':
    print ``name is john''
else if name == ``barbara'':
    print ``name is barbara''
else if name == ``eugene'':
    print ``name is eugene''
else if name == ``brett'':
    print ``name is brett''
else:
    print ``not sure what your name is''
\end{lstlisting}

With a Switch statement it can be rewritten as.

\begin{lstlisting}[caption={Switch Statement Example}]
name = ``brett''

switch name:
    case ``john'':
        print ``name is john''
        break
    case ``barbara'':
        print ``name is barbara''
        break
    case ``eugene'':
        print ``name is eugene''
        break
    case ``brett'':
        print ``name is brett''
        break
    default:
        print ``not sure what your name is''
        break
\end{lstlisting}

Both of these programs work in a similar manner, take a variable and do a direct comparison to a set of values until a match is made or else use a default action.
As well they will both output the same \pigOut{name is brett}.
Think of a Switch statement as a set of If, Else If, Else statements where the conditionals are always a single \pigVar{==}.
\par

A switch statement introduces a few new keywords, the switch followed by the variable name we wish to compare against.
Then we can have as many case statements following, each with the value that we wish to compare our variable against.
The only other weird part is that we are also introducing the break statement, which is required to terminate each case statement code block.
What the break statement says to do is ``break'' away from the entire switch statement.
As an excersise, try removing all of the break statements from the above example and run it again, what changed?
\par

We have mainly been comparing string variables against string values but you can also use Switch statements to compare numbers as well.

\begin{lstlisting}[caption={Switch Statement Numbers Example}]
age = 22

switch age:
    case 20:
        print ``not old enough to drink''
        break
    case 21:
        print ``congratulations, do not over do it''
        break
    case 22:
        print ``you`ve been doing this awhile''
        break
\end{lstlisting}

As you can see, we can also compare our number variable against number values.
In this example we also have left out the default case, this case is optional, similar to the else statement.

\subsection{For Loops}
We have seen some statements that will help the direction of our code, but what about repeating code?
Lets say that we need to manually determine what the square of a number is using multiplication (rather than the exponential operator \^).
This can be expressed fairly easily.

\begin{lstlisting}[caption={Square Without Loop}]
num = 5

newNum = num * num

print newNum
\end{lstlisting}

Fairly easy enough and we know the output of this code will be \pigOut{25}.
Now lets say we need to do this same thing but to the power of 5.

\begin{lstlisting}[caption={Power of 5 Without Loop}]
num = 5

newNum = num * num * num * num * num

print newNum
\end{lstlisting}

Ok, now this is starting to get obnoxious.
Now what if we need it to the power of 100... I'm not programming that.
This is where loops come in, in particular the For loop.
The For loop is the perfect candidate when you need an action performed a set number of times.

\begin{lstlisting}[caption={For Loop}]
num = 5

newNum = num

for i = 0; i < 99; ++i:
    newNum = newNum * num

print newNum
\end{lstlisting}

So the output of this code should be, \pigOut{7.888609052210123e+69}.
For loops can be odd to look at the first time so lets break it down part by part.
A For loop is broken into 4 parts, the Initializer, the Condition, the Update and the Code Block.
The Initializer, Condition and Update are all separated by semicolons.
\par

The Initializer, \pigVar{i = 0}, initializes some value that is going to be used throughout the loop, usually a counter; in this case \pigVar{i}.
Why is \pigVar{i} set to \pigVal{0}?
In computer programming we use a zero based counting system mainly out of tradition, but because of implementation choices made by language developers to base counting off of memory addressing offsets.
So... we just do, get in the habit now of counting from 0, everyone else does it.
\par

The Condition gets checked for every iteration of the loop and if the condition evaluates to true then the loop continues and once again executes the Code Block.
In this example \pigVar{i $<$ 99} is out Condition.
Why 99, I thought we were going to 100?
True, we are going to 100, but remember that we initially set \pigVar{newNum} to \pigVar{num} which is the same as \pigVar{num} to the first power.
Ok, so why do we use \pigVar{i $<$ 99}? won't that take us to 98?
Remember, we are using zero based counting, so 0 counts as ``1''.
\par

The Update is a statement that get executed after the Code Block and is used to update any variables we need before continuing.
In this case we are using the ``pre-increment'' operator to increase the value of \pigVar{i}, our counter, by 1.
We could have also used \pigVar{i += 1}, but \pigVar{++i} is just so elegant.
\par

Lastly, the Code Block get executed on every iteration of the loop.
The general flow of a For loop is, Initialize any variables, check the Condition if it is true then execute the Code Block, execute the Update statement, re-check the Condition, if it is true then execute the Code Block again or else leave the For loop and continue with the program.
\par

For loops are great, they save not only time, but they save a lot of typing and a lot of code duplication.
Lets say in out example above we wanted to raise 5 to the power of 50 rather than 100?
It is simple enough to change 99 to 49 and call our job done, but if we had written out \pigVar{num * num} 50 times, then it would be a pain to try and update this code.
\par

One thing to look out for with For loops, or any loops, are infinite loops, meaning a loop whose Condition will always evaluate to true.
Take the example above, if we were to change the Condition to \pigVar{i $>$= 0} then we would have an infinite loop because \pigVar{i} starts at 0 and is always increasing.
The same would be true if we changed the Condition to \pigVar{true} or \pigVar{1==1} or any other conditional statement that will always be true.

\subsection{While Loops}

\subsection{Do-While Loops}

\subsection{Break Statements}

\subsection{Continue Statements}

\subsection{Conclusion}
