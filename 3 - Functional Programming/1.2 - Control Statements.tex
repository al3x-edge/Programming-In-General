Control statements are almost exactly as they sound, statements that control our programs.
Well, they control the flow of our code.
With control statements we can change the course of our programs based on various conditions.

\subsection{If Statements}
If stataments allow us to execute a given block of code based on a given condition.
There are three main parts to an if statement \pigVal{if}, the \pigVal{conditional} and a \pigVal{code block}.


\begin{lstlisting}[caption={If Statement}]
name = ``brett''

if( name == ``brett'' )
    print ``Name is brett''
\end{lstlisting}

In this simple example the code block \pigOut{print ``Name is brett''} will only execute if the conditional \pigOut{name == ``brett''} is true.
So the output of this code will be \pigOut{Name is brett}.

\begin{lstlisting}[caption={False If Statement}]
name = ``brett''

if( name == ``john'' )
    print ``Name is john''
\end{lstlisting}

In this example there will be no output, this is because the conditional \pigOut{name == ``john''} equates to false so the code block \pigOut{print ``Name is john''} will never get executed.

\subsection{If-Else Statements}
If statements are great and help us execute portions of our code based on the values of other variables, including based on input from users.
But what if the condition of the if statement equates to false?
With if statements we can append an else statement and a block of code to the end of an if statement that will get called if the if statement is false. 


\subsection{Else If Statements}

\subsection{For Loops}

\subsection{While Loops}

\subsection{Do-While Loops}

\subsection{Switch Stataments}

\subsection{Break Statments}

\subsection{Continue Stataments}

\subsection{Conclusion}
