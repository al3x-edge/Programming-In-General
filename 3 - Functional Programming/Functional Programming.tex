In this chapter we are going to cover the basic concepts of functional programming. This could mean a few things
to different people, but in regard to this resource we are going to refer to functional programming as programming
without the use of classes and objects. Yes, some people are cringing a little in their seats as that is not the best
definition of functional programming but to try and keep things simple and organized that is what we are going to
refer to it as.
\par

I am going to use this chapter to introduce topics other than just functions. Topics including control statements, loops and some input output (io).
\par

\vfill

{\it Functional Programming:}
\\
Functional programming is a programming paradigm that treats computation as the evaluation of mathematical functions and avoids state and mutable data.
\\
Wikipedia (2012)
